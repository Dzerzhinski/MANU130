\documentclass{article}

\usepackage{amsmath}
\usepackage{enumitem}
\usepackage{array}

\begin{document}

\title{MANU 130 Lathe Project 2 Process Plan}
\author{John Bush}

\maketitle

\section*{Rough Stock}
\begin{enumerate}
	\item Select Aluminum rod, $\phi = 1.375"$, $\text{length} = 4.25"$.
\end{enumerate}

\section*{Bandsaw}
\begin{enumerate}[resume]
	\item If necessary, cut off $4.25"$ of material.
\end{enumerate}

\section*{Manual Lathe}
\begin{enumerate}[resume]
	\item Load into 3-jaw chuck.

	\item Face material.

	\item Drill through-hole in center with $\frac{1}{2}"$ twist drill.

	\item Remove from 3-jaw chuck and prepare to turn between centers.

	\item Turn outer diameter to final dimension. ($\phi = 1.313" \pm 0.010"$)

	\item Transfer to 3-jaw chuck.

	\item Face material.

	\item Bore inner diameter to final dimension. ($\phi = 0.625" \pm 0.010"$, $\text{depth} = 0.500"$)

	\item Reverse material in chuck to work opposite face.

	\item Face to final length. ($4.000 \pm 0.010"$)

	\item Bore inner diameter to final dimension. ($\phi = 0.625 \pm 0.010"$)
\end{enumerate}

\section*{Vertical Mill}

\begin{enumerate}[resume]
	\item Mount in vise with block.

	\item Mill top of part to depth $0.061"$ with $\frac{5}{8}"$ end mill.

	\item Flip over in vise to mill opposite surface to depth $0.061"$.

	\item Clamp upright in vise.

	\item Find center.  Holes are $0.339"$ on each axis ($x$ and $y$).

	\item Drill hole with \#7 twist drill to depth $0.610$.

	\item Countersink hole.  ($90^\circ$, $\text{depth} = 0.500"$ for $\phi = 0.300"$)

	\item Tap hole.  ($\frac{1}{4}$-20 UNC-1B to $\text{depth} = 0.500"$, 10 turns.)

	\item Repeat as needed for remaining three holes.

	\item Deburr as needed.
\end{enumerate}

\newpage

\begin{table}[ht]
\centering
\caption{Inspection Report}
\renewcommand\arraystretch{3}
\begin{tabular}{r | m{2cm} | m{2cm} | m{2cm} | m{2cm} | m{2cm} |}
	Dimension & Value($"$) & Max($"$) & Min($"$) & Actual($"$) & In Tol? \\
	\hline
	Length & 4.000 & 4.010 & 3.990 & & \\
	\hline
	Inner Diameter 1 & 0.625 & 0.635 & 0.615 & & \\
	\hline
	Inner Diameter 2 & 0.625 & 0.635 & 0.615 & & \\
	\hline
	Outer Diameter & 1.313 & 1.323 & 1.303 & & \\
	\hline
\end{tabular}
\end{table}

\end{document}
	
